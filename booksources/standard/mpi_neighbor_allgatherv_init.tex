% -*- latex -*-
%%%%%%%%%%%%%%%%%%%%%%%%%%%%%%%%
%%%%%%%%%%%%%%%%%%%%%%%%%%%%%%%%
%%
%% This text file is part of the source of 
%% `Parallel Computing'
%% by Victor Eijkhout, copyright 2012-2022
%%
%% MPI API file for MPI_Neighbor_allgatherv_init
%%
%% THIS FILE IS AUTO-GENERATED
%%
%%%%%%%%%%%%%%%%%%%%%%%%%%%%%%%%
%%%%%%%%%%%%%%%%%%%%%%%%%%%%%%%%

\begingroup
\ttfamily\bfseries
\catcode`\_=12
\begin{tabular}{
l       % name
l       % param name
p{\mpiparmtextsize}  % explanation
p{.7in} % ctype
p{.9in} % ftype
l       % inout
}
\toprule
\mdseries\textrm{Name}&
\mdseries\textrm{Param name}&
\mdseries\textrm{Explanation}&
\mdseries\textrm{C type}&
\mdseries\textrm{F type}&
\mdseries\textrm{inout}\\
\midrule
\hbox to 18pt{MPI_Neighbor_allgatherv_init (\hss} \\
\hbox to 18pt{MPI_Neighbor_allgatherv_init_c (\hss} \\
 & sendbuf & starting address of send buffer & const void* & TYPE(*), \hbox{}\kern10pt{}DIMENSION(..)  & IN \\ & sendcount & number of elements sent to each neighbor & $\left[ \begin{array}{ll}\mathtt{ int }  \\ \mathtt{ MPI_Count} \end{array} \right.$  & INTEGER  & IN \\ [+3pt] & sendtype & datatype of send buffer elements & MPI_Datatype & TYPE\discretionary{}{\kern10pt}{}(MPI_Datatype)  & IN \\ & recvbuf & starting address of receive buffer & void* & TYPE(*), \hbox{}\kern10pt{}DIMENSION(..)  & OUT \\ & recvcounts & non-negative integer array (of length indegree) containing the number of elements that are received from each neighbor & $\left[ \begin{array}{ll}\mathtt{ const int[] }  \\ \mathtt{ MPI_Count} \end{array} \right.$  & INTEGER\discretionary{}{\kern10pt}{}(*)  & IN \\ [+3pt] & displs & integer array (of length indegree). Entry \mpicode{i} specifies the displacement (relative to \mpiarg{recvbuf}) at which to place the incoming data from neighbor \mpicode{i} & $\left[ \begin{array}{ll}\mathtt{ const int[] }  \\ \mathtt{ MPI_Aint} \end{array} \right.$  & INTEGER\discretionary{}{\kern10pt}{}(*)  & IN \\ [+3pt] & recvtype & datatype of receive buffer elements & MPI_Datatype & TYPE\discretionary{}{\kern10pt}{}(MPI_Datatype)  & IN \\ & comm & communicator with topology structure & MPI_Comm & TYPE\discretionary{}{\kern10pt}{}(MPI_Comm)  & IN \\ & info & info argument & MPI_Info & TYPE\discretionary{}{\kern10pt}{}(MPI_Info)  & IN \\ & request & communication request & MPI_Request* & TYPE\discretionary{}{\kern10pt}{}(MPI_Request)  & OUT \\
&)\\

\bottomrule
\end{tabular}
\endgroup

