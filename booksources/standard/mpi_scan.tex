% -*- latex -*-
%%%%%%%%%%%%%%%%%%%%%%%%%%%%%%%%
%%%%%%%%%%%%%%%%%%%%%%%%%%%%%%%%
%%
%% This text file is part of the source of 
%% `Parallel Computing'
%% by Victor Eijkhout, copyright 2012-2022
%%
%% MPI API file for MPI_Scan
%%
%% THIS FILE IS AUTO-GENERATED
%%
%%%%%%%%%%%%%%%%%%%%%%%%%%%%%%%%
%%%%%%%%%%%%%%%%%%%%%%%%%%%%%%%%

\begingroup
\ttfamily\bfseries
\catcode`\_=12
\begin{tabular}{
l       % name
l       % param name
p{\mpiparmtextsize}  % explanation
p{.85in} % ctype
p{.9in} % ftype
l       % inout
}
\toprule
\mdseries\textrm{Name}&
\mdseries\textrm{Param name}&
\mdseries\textrm{Explanation}&
\mdseries\textrm{C type}&
\mdseries\textrm{F type}&
\mdseries\textrm{inout}\\
\midrule
\hbox to 18pt{MPI_Scan (\hss} \\
\hbox to 18pt{MPI_Scan_c (\hss} \\
 & sendbuf & starting address of send buffer & const\ void* & TYPE(*), \hbox{}\kern10pt{}DIMENSION(..)  & IN \\
 & recvbuf & starting address of receive buffer & void* & TYPE(*), \hbox{}\kern10pt{}DIMENSION(..)  & OUT \\
 & count & number of elements in input buffer & $\left[ \begin{array}{ll}\mathtt{ int }  \\ \mathtt{ MPI_Count} \end{array} \right.$  & INTEGER  & IN \\ [+3pt]
 & datatype & datatype of elements of input buffer & MPI_Datatype & TYPE\discretionary{}{\kern10pt}{}(MPI_Datatype)  & IN \\
 & op & operation & MPI_Op & TYPE\discretionary{}{\kern10pt}{}(MPI_Op)  & IN \\
 & comm & communicator & MPI_Comm & TYPE\discretionary{}{\kern10pt}{}(MPI_Comm)  & IN \\

&)\\

\bottomrule
\end{tabular}
\endgroup

