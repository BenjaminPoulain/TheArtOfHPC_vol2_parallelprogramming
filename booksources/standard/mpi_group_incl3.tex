% -*- latex -*-
%%%%%%%%%%%%%%%%%%%%%%%%%%%%%%%%
%%%%%%%%%%%%%%%%%%%%%%%%%%%%%%%%
%%
%% This text file is part of the source of 
%% `Parallel Computing'
%% by Victor Eijkhout, copyright 2012-2022
%%
%% MPI API file for MPI_Group_incl
%%
%% THIS FILE IS AUTO-GENERATED
%%
%%%%%%%%%%%%%%%%%%%%%%%%%%%%%%%%
%%%%%%%%%%%%%%%%%%%%%%%%%%%%%%%%

\begingroup
\ttfamily\bfseries
\catcode`\_=12
\begin{tabular}{
l       % name
l       % param name
p{\mpiparmtextsize}  % explanation
p{.85in} % ctype
p{.9in} % ftype
l       % inout
}
\toprule
\mdseries\textrm{Name}&
\mdseries\textrm{Param name}&
\mdseries\textrm{Explanation}&
\mdseries\textrm{C type}&
\mdseries\textrm{F type}&
\mdseries\textrm{inout}\\
\midrule
\hbox to 18pt{MPI_Group_incl (\hss} \\
 & group & group & MPI_Group & TYPE\discretionary{}{\kern10pt}{}(MPI_Group)  \\
 & n & number of elements in array \mpiarg{ranks} (and size of \mpiarg{newgroup}) & int & INTEGER  \\
 & ranks & ranks of processes in \mpiarg{group} to appear in \mpiarg{newgroup} & const\ int[] & INTEGER\discretionary{}{\kern10pt}{}(n)  \\
 & newgroup & new group derived from above, in the order defined by \mpiarg{ranks} & MPI_Group* & TYPE\discretionary{}{\kern10pt}{}(MPI_Group)  \\

&)\\

\bottomrule
\end{tabular}
\endgroup

