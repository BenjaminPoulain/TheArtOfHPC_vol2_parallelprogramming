% -*- latex -*-
%%%%%%%%%%%%%%%%%%%%%%%%%%%%%%%%
%%%%%%%%%%%%%%%%%%%%%%%%%%%%%%%%
%%
%% This text file is part of the source of 
%% `Parallel Computing'
%% by Victor Eijkhout, copyright 2012-2022
%%
%% MPI API file for MPI_Ialltoallw
%%
%% THIS FILE IS AUTO-GENERATED
%%
%%%%%%%%%%%%%%%%%%%%%%%%%%%%%%%%
%%%%%%%%%%%%%%%%%%%%%%%%%%%%%%%%

\begingroup
\ttfamily\bfseries
\catcode`\_=12
\begin{tabular}{
l       % name
l       % param name
p{\mpiparmtextsize}  % explanation
p{.85in} % ctype
p{.9in} % ftype
l       % inout
}
\toprule
\mdseries\textrm{Name}&
\mdseries\textrm{Param name}&
\mdseries\textrm{Explanation}&
\mdseries\textrm{C type}&
\mdseries\textrm{F type}&
\mdseries\textrm{inout}\\
\midrule
\hbox to 18pt{MPI_Ialltoallw (\hss} \\
\hbox to 18pt{MPI_Ialltoallw_c (\hss} \\
 & sendbuf & starting address of send buffer & const\ void* & TYPE(*), \hbox{}\kern10pt{}DIMENSION(..)  & IN \\
 & sendcounts & integer array (of length group size) specifying the number of elements to send to each rank & $\left[ \begin{array}{ll}\mathtt{ const\ int[] }  \\ \mathtt{ MPI_Count} \end{array} \right.$  & INTEGER\discretionary{}{\kern10pt}{}(*)  & IN \\ [+3pt]
 & sdispls & integer array (of length group size). Entry \mpicode{j} specifies the displacement in bytes (relative to \mpiarg{sendbuf}) from which to take the outgoing data destined for process \mpicode{j} & $\left[ \begin{array}{ll}\mathtt{ const\ int[] }  \\ \mathtt{ MPI_Aint} \end{array} \right.$  & INTEGER\discretionary{}{\kern10pt}{}(*)  & IN \\ [+3pt]
 & sendtypes & array of datatypes (of length group size). Entry \mpicode{j} specifies the type of data to send to process \mpicode{j} & const\ MPI_Datatype[] & TYPE\discretionary{}{\kern10pt}{}(MPI_Datatype)\discretionary{}{\kern10pt}{}(*)  & IN \\
 & recvbuf & address of receive buffer & void* & TYPE(*), \hbox{}\kern10pt{}DIMENSION(..)  & OUT \\
 & recvcounts & integer array (of length group size) specifying the number of elements that can be received from each rank & $\left[ \begin{array}{ll}\mathtt{ const\ int[] }  \\ \mathtt{ MPI_Count} \end{array} \right.$  & INTEGER\discretionary{}{\kern10pt}{}(*)  & IN \\ [+3pt]
 & rdispls & integer array (of length group size). Entry \mpicode{i} specifies the displacement in bytes (relative to \mpiarg{recvbuf}) at which to place the incoming data from process \mpicode{i} & $\left[ \begin{array}{ll}\mathtt{ const\ int[] }  \\ \mathtt{ MPI_Aint} \end{array} \right.$  & INTEGER\discretionary{}{\kern10pt}{}(*)  & IN \\ [+3pt]
 & recvtypes & array of datatypes (of length group size). Entry \mpicode{i} specifies the type of data received from process \mpicode{i} & const\ MPI_Datatype[] & TYPE\discretionary{}{\kern10pt}{}(MPI_Datatype)\discretionary{}{\kern10pt}{}(*)  & IN \\
 & comm & communicator & MPI_Comm & TYPE\discretionary{}{\kern10pt}{}(MPI_Comm)  & IN \\
 & request & communication request & MPI_Request* & TYPE\discretionary{}{\kern10pt}{}(MPI_Request)  & OUT \\

&)\\

\bottomrule
\end{tabular}
\endgroup

