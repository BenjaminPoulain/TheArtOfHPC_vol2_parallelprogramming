% -*- latex -*-
%%%%%%%%%%%%%%%%%%%%%%%%%%%%%%%%%%%%%%%%%%%%%%%%%%%%%%%%%%%%%%%%
%%%%%%%%%%%%%%%%%%%%%%%%%%%%%%%%%%%%%%%%%%%%%%%%%%%%%%%%%%%%%%%%
%%%%
%%%% This text file is part of the source of 
%%%% `Parallel Programming in MPI and OpenMP'
%%%% by Victor Eijkhout, copyright 2012-2022
%%%%
%%%% projectstyle.tex : guidelins for project writeup
%%%%
%%%%%%%%%%%%%%%%%%%%%%%%%%%%%%%%%%%%%%%%%%%%%%%%%%%%%%%%%%%%%%%%
%%%%%%%%%%%%%%%%%%%%%%%%%%%%%%%%%%%%%%%%%%%%%%%%%%%%%%%%%%%%%%%%

Here are some guidelines for how to submit assignments and projects.
As a general rule, consider programming as an experimental science,
and your writeup as a report on some tests you have done: explain
the problem you're addressing, your strategy, your results.

Turn in a writeup in pdf form (Word and text documents are not acceptable)
that was generated from a text processing program such 
(preferably) \LaTeX\ (for a tutorial, see~\CARPref{tut:latex}).

\Level 0 {Structure of your writeup}

\Level 1 {Write as if it's an article}

Consider this project writeup an opportunity to practice writing a scientific article.

Start with the obvious stuff.
\begin{itemize}
\item Your writeup should have a title. Not `Project' or `parallel programming',
  but something like 'Parallelization of Chronosynclastic Enfundibula in MPI'.
\item Author and contact information. This differs per publication.
  Here it is: your name, EID, TACC username, and email.
\item Introductory section that is extremely high level: what is the problem,
  what did you do, what did you find.
\item Conclusion: what do your findings mean, what are limitations, opportunities
  for future extensions.
\item Bibliography.
\end{itemize}

\Level 1 {Consider your audience}

An article is written for a specific audience: a journal,  a conference,
or in this case: your instructors.
So don't go into details that mean nothing to your audience, and
try giving them what they find interesting.

In other words: give enough background on your application, but not too much.
You're not writing for your thesis supervisor, you're writing to
interested outsiders to your field.

On the other hand, your instructors know everything about parallelism.
So don't show a differential equation and say `and I made this parallel with OpenMP'.
Go into detail how you translated your problem into something computational,
and then show relevant bits of code.

That does not mean that turning in the code is sufficient, nor code plus sample output.
Write an article.

\Level 1 {Observe, measure, hypothesize, deduce}

Your project may be a scientific investigation of some
phenomenon. Formulate hypotheses as to what you expect to observe,
report on your observations, and draw conclusions.

Quite often your program will display unexpected behaviour. It is important to observe
this, and hypothesize what the reason might be for your observed behaviour.

In most applications of computing machinery we care about the efficiency with which
we find the solution. Thus, make sure that you do measurements. In general, make
observations that allow you to judge whether your program behaves the way you
would expect it to.

\Level 1 {Reporting}

Include both code snippets and graphs.

Screenshots of code snippets are not acceptable.
Use at least a verbatim/monospace mode in your text processor,
but better, use the \LaTeX\ \indextermtt{listings} package or equivalent.

Graphs can be generated any number of ways.
Kudos if you can figure out the \LaTeX\ \indextermtt{tikz} package,
but Matlab or Excel are acceptable too.
No screenshots though.

For parallel runs you can, but are not required to,
use TAU plots; see~\CARPref{sec:tau}.

\Level 1 {Repository organization}

If you submit your work through a repository
have your pdf file at the top level;
organize your sources in clearly named subdirectories.
Object files and binaries should not be in a repository
since they are dependent on hardware and things like compilers.

\Level 0 {The parallel part}

The parallelization part is the most important of your writeup.
So don't write 3 pages about your application and 1~about the parallel code.
Discuss in detail:
\begin{itemize}
\item Did you use MPI or OpenMP? Why?
\item What is the parallel structure of your problem? What does the parallel work correspond
  to in terms of your application?
\item What kind of parallelism did you use? Mostly MPI collectives or point-to-point operations?
  OpenMP loop parallelism or tasks? Why?
\end{itemize}

\Level 1 {Running your code}

A single run doesn't prove anything. For a good report, you need to
run your code for more than one input dataset (if available) and in
more than one processor configuration. When you choose problem sizes,
be aware that an average processor can do a billion operations per
second: you need to make your problem large enough for the timings to
rise above the level of random variations and startup phenomena.

When you run a code in parallel, beware that on clusters the behaviour
of a parallel code will always be different between one node and
multiple nodes.  On a single node the MPI implementation is likely
optimized to use the shared memory. This means that results obtained
from a single node run will be unrepresentative. In fact, in timing
and scaling tests you will often see a drop in (relative) performance
going from one node to two.  Therefore you need to run your code in a
variety of scenarios, using more than one node.

\Level 1 {Graphs}

In parallel programming, speedup and scaling are the test of
how good your work is.
So it's up to you to report this as well as you can.

If you do a scaling analysis, a graph reporting runtimes should not
have a linear time axis.
(Curved graphs are hard to read. Can you see the difference between
$O(\sqrt n)$ and $O(\log n)$ behavior in a graph?)
Try to find a way
to compare your results to a straight line, such as constant time, or
linearly increasing performance.

It is up to you to decide what quantity to report.
This may depend on your application.

Use enough data points! Writing a short script to run
your program multiple times takes very little time.

\Level 0 {Helpful remarks}

\Level 1 {Parallel performance or the lack thereof}

In a perfect world, the performance of your code should grow
with the number of available resources.
If your program shows disappointing performance,
consider the following.

Synchronizing OpenMP threads at the end of a parallel region takes
maybe a few hundred cycles. This means that the amount of work
in that region should be considerably more.

If your OpenMP program stops scaling at a certain core count,
consider affinity settings; section~\ref{sec:omp-proc-bind}.

MPI messages takes a couple of microseconds. Again, this implies that the
amount of work between messages needs to be large enough.

\Level 1 {Code formatting}

Included code snippets should be readable. At a minimum you could 
indent the code correctly in an editor before you include it in
a \indextermtt{verbatim} environment. (Screenshots of your terminal
window are a decidedly suboptimal solution.)
But it's better to use the \indextermtt{listing} package which formats
your code, include syntax coloring. For instance, 
\begin{verbatim}
\lstset{language=C++} % or Fortran or so
\begin{lstlisting}
for (int i=0; i<N; i++)
  s += 1;
\end{lstlisting}
\end{verbatim}
\lstset{language=C++} % or Fortran or so
\begin{lstlisting}
for (int i=0; i<N; i++)
  s += 1;
\end{lstlisting}

\endinput

\Level 1 {Including code in your writeup}

If you include code samples in your writeup, make sure they look good. For starters,
use a mono-spaced font. In \LaTeX, you can use the \n{verbatim} environment or the 
\n{verbatiminput} command. In that section option the source is included automatically,
rather than cut and pasted. This is to be preferred, since your writeup will
stay current after you edit the source file.

Including whole source files makes for a long and boring writeup. The code samples in this
book were generated as follows. In the source files, the relevant snippet was marked as
\begin{verbatim}
... boring stuff
//snippet samplex
  .. interesting! ..
//snippet end
... more boring stuff
\end{verbatim}
The files were then processed with the following command line (actually, included
in a makefile, which requires doubling the dollar signs):
\begin{verbatim}
for f in *.{c,cxx,h} ; do
  cat $x | awk 'BEGIN {f=0}
                /snippet end/ {f=0}
                f==1 {print $0 > file}
                /snippet/ && !/end/ {f=1; file=$2 }
               '
done
\end{verbatim}
which gives (in this example) a file \n{samplex}. Other solutions are of course possible.

