% -*- latex -*-
%%%%%%%%%%%%%%%%%%%%%%%%%%%%%%%%%%%%%%%%%%%%%%%%%%%%%%%%%%%%%%%%
%%%%%%%%%%%%%%%%%%%%%%%%%%%%%%%%%%%%%%%%%%%%%%%%%%%%%%%%%%%%%%%%
%%%%
%%%% This text file is part of the source of 
%%%% `Parallel Programming in MPI and OpenMP'
%%%% by Victor Eijkhout, copyright 2012-9
%%%%
%%%% omp-simd.tex : SIMD instruction processing
%%%%
%%%%%%%%%%%%%%%%%%%%%%%%%%%%%%%%%%%%%%%%%%%%%%%%%%%%%%%%%%%%%%%%
%%%%%%%%%%%%%%%%%%%%%%%%%%%%%%%%%%%%%%%%%%%%%%%%%%%%%%%%%%%%%%%%

You can declare a loop to be executable with
\indextermbus{vector}{instructions} with
\indexpragmadef{simd}.

\begin{remark}
  Depending on your compiler, it may be necessary to give an extra option
  enabling SIMD:
  \begin{itemize}
  \item  \n{-fopenmp-simd} for \indexterm{GCC}~/ \indexterm{Clang}, and
  \item \n{-qopenmp-simd} for \indexterm{ICC}.
  \end{itemize}
\end{remark}

The \indexpragma{simd} pragma has the following clauses:
\begin{itemize}
\item \indexclause{safelen($n$)}: limits the number of iterations in a
  SIMD chunk. Presumably useful if you combine \n{parallel for simd}.
\item \indexclause{linear}: lists variables that have a linear
  relation to the iteration parameter.
\item \indexclause{aligned}: specifies alignment of variables.
\end{itemize}

If your SIMD loop includes a function call, you can declare that the
function can be turned into vector instructions with
%
\indexpragma{declare simd}

If a loop is both multi-threadable and vectorizable, you can combine
directives as \n{pragma omp parallel for simd}.

Compilers can be made to report whether a loop was vectorized:
\begin{verbatim}
   LOOP BEGIN at simdf.c(61,15)
      remark #15301: OpenMP SIMD LOOP WAS VECTORIZED
   LOOP END
\end{verbatim}
with such options as \n{-Qvec-report=3} for the Intel compiler.

Performance improvements of these directives need not be immediately
obvious.  In cases where the operation is bandwidth-limited, using
\n{simd} parallelism may give the same or worse performance as thread
parallelism.

The following function can be vectorized:
%
\cverbatimsnippet[code/omp/c/simd/tools.c]{simdfunction}
%
Compiling this the regular way
\begin{verbatim}
# parameter 1(x1): %xmm0
# parameter 2(x2): %xmm1
# parameter 3(y1): %xmm2
# parameter 4(y2): %xmm3

movaps    %xmm0, %xmm5    5 <- x1
movaps    %xmm2, %xmm4    4 <- y1
mulsd     %xmm1, %xmm5    5 <- 5 * x2 = x1 * x2
mulsd     %xmm3, %xmm4    4 <- 4 * y2 = y1 * y2
mulsd     %xmm0, %xmm0    0 <- 0 * 0 = x1 * x1
mulsd     %xmm1, %xmm1    1 <- 1 * 1 = x2 * x2
addsd     %xmm4, %xmm5    5 <- 5 + 4 = x1*x2 + y1*y2
mulsd     %xmm2, %xmm2    2 <- 2 * 2 = y1 * y1
mulsd     %xmm3, %xmm3    3 <- 3 * 3 = y2 * y2
addsd     %xmm1, %xmm0    0 <- 0 + 1 = x1*x1 + x2*x2
addsd     %xmm3, %xmm2    2 <- 2 + 3 = y1*y1 + y2*y2
sqrtsd    %xmm0, %xmm0    0 <- sqrt(0) = sqrt( x1*x1 + x2*x2 )
sqrtsd    %xmm2, %xmm2    2 <- sqrt(2) = sqrt( y1*y1 + y2*y2 )
\end{verbatim}
which uses the scalar instruction \indextermtt{mulsd}: multiply scalar double precision.

With a \n{declare simd} directive:
\begin{verbatim}
movaps    %xmm0, %xmm7
movaps    %xmm2, %xmm4
mulpd     %xmm1, %xmm7
mulpd     %xmm3, %xmm4
\end{verbatim}
which uses the vector instruction \indextermtt{mulpd}: multiply packed
double precision, operating on 128-bit \indextermsub{SSE2}{register}s.

%% make tools.o MOREOPTIONS="-Fa -xMIC-AVX512"

Compiling for the \indextermbus{Intel}{Knight's Landing} gives more complicated code:
\begin{verbatim}
# parameter 1(x1): %xmm0
# parameter 2(x2): %xmm1
# parameter 3(y1): %xmm2
# parameter 4(y2): %xmm3

vmulpd    %xmm3, %xmm2, %xmm4                           4 <- y1*y2
vmulpd    %xmm1, %xmm1, %xmm5                           5 <- x1*x2
vbroadcastsd .L_2il0floatpacket.0(%rip), %zmm21         
movl      $3, %eax                                      set accumulator EAX
vbroadcastsd .L_2il0floatpacket.5(%rip), %zmm24         
kmovw     %eax, %k3                                     set mask k3
vmulpd    %xmm3, %xmm3, %xmm6                           6 <-y1*y1 (stall)
vfmadd231pd %xmm0, %xmm1, %xmm4                         4 <- 4 + x1*x2 (no reuse!)
vfmadd213pd %xmm5, %xmm0, %xmm0                         0 <- 0 + 0*5 = x1 + x1*(x1*x2)
vmovaps   %zmm21, %zmm18                                #25.26 c7
vmovapd   %zmm0, %zmm3{%k3}{z}                          #25.26 c11
vfmadd213pd %xmm6, %xmm2, %xmm2                         #24.29 c13
vpcmpgtq  %zmm0, %zmm21, %k1{%k3}                       #25.26 c13
vscalefpd .L_2il0floatpacket.1(%rip){1to8}, %zmm0, %zmm3{%k1} #25.26 c15
vmovaps   %zmm4, %zmm26                                 #25.26 c15
vmovapd   %zmm2, %zmm7{%k3}{z}                          #25.26 c17
vpcmpgtq  %zmm2, %zmm21, %k2{%k3}                       #25.26 c17
vscalefpd .L_2il0floatpacket.1(%rip){1to8}, %zmm2, %zmm7{%k2} #25.26 c19
vrsqrt28pd %zmm3, %zmm16{%k3}{z}                        #25.26 c19
vpxorq    %zmm4, %zmm4, %zmm26{%k3}                     #25.26 c19
vrsqrt28pd %zmm7, %zmm20{%k3}{z}                        #25.26 c21
vmulpd    {rn-sae}, %zmm3, %zmm16, %zmm19{%k3}{z}       #25.26 c27 stall 2
vscalefpd .L_2il0floatpacket.2(%rip){1to8}, %zmm16, %zmm17{%k3}{z} #25.26 c27
vmulpd    {rn-sae}, %zmm7, %zmm20, %zmm23{%k3}{z}       #25.26 c29
vscalefpd .L_2il0floatpacket.2(%rip){1to8}, %zmm20, %zmm22{%k3}{z} #25.26 c29
vfnmadd231pd {rn-sae}, %zmm17, %zmm19, %zmm18{%k3}      #25.26 c33 stall 1
vfnmadd231pd {rn-sae}, %zmm22, %zmm23, %zmm21{%k3}      #25.26 c35
vfmadd231pd {rn-sae}, %zmm19, %zmm18, %zmm19{%k3}       #25.26 c39 stall 1
vfmadd231pd {rn-sae}, %zmm23, %zmm21, %zmm23{%k3}       #25.26 c41
vfmadd213pd {rn-sae}, %zmm17, %zmm17, %zmm18{%k3}       #25.26 c45 stall 1
vfnmadd231pd {rn-sae}, %zmm19, %zmm19, %zmm3{%k3}       #25.26 c47
vfmadd213pd {rn-sae}, %zmm22, %zmm22, %zmm21{%k3}       #25.26 c51 stall 1
vfnmadd231pd {rn-sae}, %zmm23, %zmm23, %zmm7{%k3}       #25.26 c53
vfmadd213pd %zmm19, %zmm18, %zmm3{%k3}                  #25.26 c57 stall 1
vfmadd213pd %zmm23, %zmm21, %zmm7{%k3}                  #25.26 c59
vscalefpd .L_2il0floatpacket.3(%rip){1to8}, %zmm3, %zmm3{%k1} #25.26 c63 stall 1
vscalefpd .L_2il0floatpacket.3(%rip){1to8}, %zmm7, %zmm7{%k2} #25.26 c65
vfixupimmpd $112, .L_2il0floatpacket.4(%rip){1to8}, %zmm0, %zmm3{%k3} #25.26 c65
vfixupimmpd $112, .L_2il0floatpacket.4(%rip){1to8}, %zmm2, %zmm7{%k3} #25.26 c67
vmulpd    %xmm7, %xmm3, %xmm0                           #25.26 c71
vmovaps   %zmm0, %zmm27                                 #25.26 c79
vmovaps   %zmm0, %zmm25                                 #25.26 c79
vrcp28pd  {sae}, %zmm0, %zmm27{%k3}                     #25.26 c81
vfnmadd213pd {rn-sae}, %zmm24, %zmm27, %zmm25{%k3}      #25.26 c89 stall 3
vfmadd213pd {rn-sae}, %zmm27, %zmm25, %zmm27{%k3}       #25.26 c95 stall 2
vcmppd    $8, %zmm26, %zmm27, %k1{%k3}                  #25.26 c101 stall 2
vmulpd    %zmm27, %zmm4, %zmm1{%k3}{z}                  #25.26 c101
kortestw  %k1, %k1                                      #25.26 c103
je        ..B1.3        # Prob 25%                      #25.26 c105
vdivpd    %zmm0, %zmm4, %zmm1{%k1}                      #25.26 c3 stall 1
vmovaps   %xmm1, %xmm0                                  #25.26 c77
ret                                                     #25.26 c79
\end{verbatim}

\begin{lstlisting}
#pragma omp declare simd uniform(op1) linear(k) notinbranch
  double SqrtMul(double *op1, double op2, int k) {
    return (sqrt(op1[k]) * sqrt(op2));
  }
\end{lstlisting}
