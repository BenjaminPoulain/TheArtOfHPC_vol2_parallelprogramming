% -*- latex -*-
%%%%%%%%%%%%%%%%%%%%%%%%%%%%%%%%%%%%%%%%%%%%%%%%%%%%%%%%%%%%%%%%
%%%%%%%%%%%%%%%%%%%%%%%%%%%%%%%%%%%%%%%%%%%%%%%%%%%%%%%%%%%%%%%%
%%%%
%%%% This text file is part of the source of 
%%%% `Parallel Programming in MPI and OpenMP'
%%%% by Victor Eijkhout, copyright 2012-2021
%%%%
%%%% mpi-competency.tex : goals of the MPI part
%%%%
%%%%%%%%%%%%%%%%%%%%%%%%%%%%%%%%%%%%%%%%%%%%%%%%%%%%%%%%%%%%%%%%
%%%%%%%%%%%%%%%%%%%%%%%%%%%%%%%%%%%%%%%%%%%%%%%%%%%%%%%%%%%%%%%%

This section of the book teaches MPI (`Message Passing Interface'),
the dominant model for distributed memory programming in science and engineering.
It will instill the following competencies.

Basic level:
\begin{itemize}
\item The student will understand the \acs{SPMD} model
  and how it is realized in MPI (chapter~\ref{ch:mpi-functional}).
\item The student will know the basic collective calls,
  both with and without a root process,
  and can use them in applications
  (chapter~\ref{ch:mpi-collective}).
\item The student knows the basic blocking and non-blocking
  point-to-point calls,
  and how to use them
  (chapter~\ref{ch:mpi-ptp}).
\end{itemize}

Intermediate level:
\begin{itemize}
\item The students knows about derived datatypes and can use them
  in communication routines
  (chapter~\ref{ch:mpi-data}).
\item The student knows about intra-communicators,
  and some basic calls for creating subcommunicators
  (chapter~\ref{ch:mpi-comm});
  also Cartesian process topologies
  (section~\ref{sec:cartesian}).
\item The student understands the basic design of MPI I/O calls
  and can use them in basic applications
  (chapter~\ref{ch:mpi-io}).
\item The student understands about graph process topologies
  and neighborhood collectives
  (section~\ref{sec:mpi-dist-graph}).
\end{itemize}

Advanced level:
\begin{itemize}
\item The student understands one-sided communication routines,
  including window creation routines, and synchronization mechanisms
  (chapter~\ref{ch:mpi-onesided}).
\item The student understands MPI shared memory, its advantages,
  and ow it is based on windows
  (chapter~\ref{ch:mpi-shared}).
\item The student understands MPI process spawning mechanisms
  and inter-communicators
  (chapter~\ref{ch:mpi-proc}).
%\item  (chapter~\ref{ch:mpi-}).
\end{itemize}
