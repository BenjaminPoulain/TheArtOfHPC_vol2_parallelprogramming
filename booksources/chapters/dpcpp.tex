% -*- latex -*-
%%%%%%%%%%%%%%%%%%%%%%%%%%%%%%%%%%%%%%%%%%%%%%%%%%%%%%%%%%%%%%%%
%%%%%%%%%%%%%%%%%%%%%%%%%%%%%%%%%%%%%%%%%%%%%%%%%%%%%%%%%%%%%%%%
%%%%
%%%% This text file is part of the source of 
%%%% `Parallel Programming in MPI and OpenMP'
%%%% by Victor Eijkhout, copyright 2012-2021
%%%%
%%%% dpcpp.tex : intro stuff about DPC++
%%%%
%%%%%%%%%%%%%%%%%%%%%%%%%%%%%%%%%%%%%%%%%%%%%%%%%%%%%%%%%%%%%%%%
%%%%%%%%%%%%%%%%%%%%%%%%%%%%%%%%%%%%%%%%%%%%%%%%%%%%%%%%%%%%%%%%

This chapter explains the basic concepts of Sycl/Dpc++,
and helps you get
started on running your first program.

\begin{itemize}
\item \indextermdef{SYCL} is a C++-based language for portable parallel programming.
\item \indexac{DPCPP} is Intel's extension of Sycl.
\item \indextermdef{OneAPI} is Intel's compiler suite,
  which contains the \ac{DPCPP} compiler.
\end{itemize}

\begin{dpcppnote}
  The various Intel extensions are listed here:
  \url{https://spec.oneapi.com/versions/latest/elements/dpcpp/source/index.html#extensions-table}
\end{dpcppnote}

\Level 0 {Logistics}

Headers:
\begin{lstlisting}
#include <CL/sycl.hpp>
\end{lstlisting}

You can now include namespace, but with care!
If you use
\begin{lstlisting}
using namespace cl;
\end{lstlisting}
you have to prefix all SYCL class with \lstinline+sycl::+,
which is a bit of a bother.
However, if you use
\begin{lstlisting}
using namespace cl::sycl;
\end{lstlisting}
you run into the fact that SYCL has its own versions of many \ac{STL}
commands, and so you will get name collisions.
The most obvious example is that 
the \lstinline+cl::sycl+ name space has its own versions of \n{cout} and \n{endl}.
Therefore you have to use explicitly \lstinline+std::cout+ and \lstinline+std::end+.
Using the wrong I/O will cause tons of inscrutable error messages.
Additionally, SYCL has its own version of \lstinline{free},
and of several math routines.

\begin{dpcppnote}
  \begin{lstlisting}
    using namespace sycl;
  \end{lstlisting}
\end{dpcppnote}

\Level 0 {Platforms and devices}

Since \ac{DPCPP} is cross-platform, we first need to discovers
the devices.

First we list the platforms:
\cxxverbatimsnippet[code/dpcpp/cxx/devices.cxx]{dpcppplatforms}

Then for each platform we list the devices:
\cxxverbatimsnippet[code/dpcpp/cxx/devices.cxx]{dpcppdevices}

You can query what type of device you are dealing with by
\indexsyclshow{is_host}, \indexsyclshow{is_cpu}, \indexsyclshow{is_gpu}.

\Level 0 {Queues}

The execution mechanism of SYCL is the
\emph{queue}\index{queue!SYCL}:
a sequence of actions that will be executed on a selected device.
The only user action is submitting actions to a queue;
the queue is executed at the end of the scope where it is declared.

Queue execution is asynchronous with host code.

\Level 1 {Device selectors}

You need to select a device on which to execute the queue.
A~single queue can only dispatch to a single device.

A queue is coupled to one specific device,
so it can not spread work over multiple devices.
You can find a default device for the queue with
\begin{lstlisting}
  sycl::queue myqueue;
\end{lstlisting}

The following example explicitly assigns the queue to the CPU device
using the \lstinline+sycl::+\indexsyclshow{cpu_selector}.
\cxxverbatimsnippet[code/dpcpp/cxx/cpuname.cxx]{cpuclass}

The \lstinline+sycl::+\indexsyclshow{host_selector} bypasses any devices and
make the code run on the host.

It is good for your sanity to print the name of the device
you are running on:
\cxxverbatimsnippet[code/dpcpp/cxx/devname.cxx]{devname}

If you try to select a device that is not available,
a \lstinline+sycl::+\indexsyclshow{runtime_error} exception will be thrown.

\begin{dpcppnote}
  \begin{lstlisting}
    #include "CL/sycl/intel/fpga_extensions.hpp"
    fpga_selector
  \end{lstlisting}
\end{dpcppnote}

\Level 1 {Queue submission and execution}

It seems that queue kernels will also be executed when only they
go out of scope, but not the queue:
%
\cxxverbatimsnippet{dpcppsubmitbuf}

\Level 1 {Kernel ordering}

Kernels are not necessarily executed in the order in which they are submitted.
You can enforce this by specifying an \indextermsub{in-order}{queue}:
\begin{lstlisting}
sycl::queue myqueue{property::queue::inorder()};
\end{lstlisting}

\Level 0 {Kernels}

One kernel per submit.

\begin{lstlisting}
myqueue.submit( [&] ( handler &commandgroup ) {
    commandgroup.parallel_for<uniquename> 
      ( range<1>{N},
        [=] ( id<1> idx ) { ... idx }
      )
    } );
\end{lstlisting}

Note that the lambda in the kernel captures by value.
Capturing by reference makes no sense,
since the kernel is executed on a device.

\begin{lstlisting}
cgh.single_task(
  [=]() {
    // kernel function is executed EXACTLY once on a SINGLE work-item
});
\end{lstlisting}

The \indexsyclshow{submit} call results in an event object:
\begin{lstlisting}
auto myevent = myqueue.submit( /* stuff */ );
\end{lstlisting}
This can be used for two purposes:
\begin{enumerate}
\item It becomes possible to wait for this specific event:
\begin{lstlisting}
myevent.wait();    
\end{lstlisting}
\item It can be used to indicate kernel dependencies:
\begin{lstlisting}
myqueue.submit( [=] (handler &h) {
    h.depends_on(myevent);
    /* stuff */
    } );
\end{lstlisting}
\end{enumerate}

\Level 0 {Parallel operations}

\Level 1 {Loops}
  
\begin{lstlisting}
cgh.parallel_for(
  range<3>(1024,1024,1024),
  // using 3D in this example
  [=](id<3> myID) {
    // kernel function is executed on an n-dimensional range (NDrange)
});

cgh.parallel_for(
  nd_range<3>( {1024,1024,1024},{16,16,16} ),
  // using 3D in this example 
  [=](nd_item<3> myID) {
    // kernel function is executed on an n-dimensional range (NDrange)
});

cgh.parallel_for_work_group(
  range<2>(1024,1024),
  // using 2D in this example
  [=](group<2> myGroup) {
    // kernel function is executed once per work-group
});

grp.parallel_for_work_item(
  range<1>(1024),
  // using 1D in this example
  [=](h_item<1> myItem) {
    // kernel function is executed once per work-item
});
\end{lstlisting}

\Level 2 {Loop bounds: ranges}

SYCL adopts the modern C++ philosophy that one does not iterate
over by explicitly enumerating indices, but by indicating their range.
This is realized by the \indexsycldef{range} class,
which is templated over the number of space dimensions.

\begin{lstlisting}
sycl::range<2> matrix{10,10};
\end{lstlisting}

Some compilers are sensitive to the type of the integer arguments:
\begin{lstlisting}
sycl::range<1> array{ static_cast<size_t>(size)} ;
\end{lstlisting}

\Level 2 {Loop indices}

Kernels such as \indexsyclshow{parallel_for}
expects two arguments:
\begin{itemize}
\item a \indexsyclshow{range} over which to index; and
\item a lambda of one argument: an index.
\end{itemize}

There are several ways of indexing.
The \indexsyclshow{id<nd>} class of multi-dimensional indices.
\begin{lstlisting}
myHandle.parallel_for<class uniqueID>
   ( mySize,
     [=]( id<1> index ) {
       float x = index.get(0) * h;
       deviceAccessorA[index] *= 2.;
     }
   )
\end{lstlisting}

\begin{lstlisting}
cgh.parallel_for<class foo>(
    range<1>{D*D*D},
    [=](id<1> item) {
        xx[ item[0] ] = 2 * item[0] + 1;
    }
)
\end{lstlisting}

While the C++ vectors remain one-dimensional,
\ac{DPCPP} allows you to make multi-dimensional buffers:
\begin{lstlisting}
std::vector<int> y(D*D*D);
buffer<int,1> y_buf(y.data(), range<1>(D*D*D));
cgh.parallel_for<class foo2D>
   (range<2>{D,D*D},
    [=](id<2> item) {
        yy[ item[0] + D*item[1] ] = 2;
    }
   );
\end{lstlisting}

\begin{dpcppnote}
  There is an implicit conversion from the one-dimensional
  \lstinline+sycl::+\indexsyclshow{id<1>}
  to \lstinline+size_t+, so
\begin{lstlisting}
[=](sycl::id<1> i) {
   data[i] = i;
}
\end{lstlisting}
is legal, which in SYCL requires
\begin{lstlisting}
data[i[0]] = i[0];
\end{lstlisting}
\end{dpcppnote}

\Level 2 {Multi-dimensional indexing}

\cxxverbatimsnippet{syclrange2}

\cxxverbatimsnippet{syclrange2nd}

We first copy global data into an array local to the work group:
%
\cxxverbatimsnippet{syclnduse}

Global coordinates in the input are computed from the
 \indexsyclshow{nd_item}'s coordinate and group:
\begin{lstlisting}
[=] ( sycl::nd_item<2> it ) {
for   (int ti ... ) {
  for (int tj ... ) {
    int gi = ti + B * it.get_group(0);
    int gj = tj + B * it.get_group(1);
    ... = input[gi][gj];
\end{lstlisting}

Local coordinates in the tile, including boundary,
I DON'T QUITE GET THIS YET.
\begin{lstlisting}
[=] ( sycl::nd_item<2> it ) {
sycl::id<2>    lid    = it.get_local_id();
sycl::range<2> lrange = it.get_local_range();
for   (int ti = lid[0]; ti < B + 2; ti += lrange[0]) {
  for (int tj = lid[1]; tj < B + 2; tj += lrange[1]) {
    tile[ti][tj] = ..
\end{lstlisting}

\Level 1 {Task dependencies}

Each \indexmplshow{submit} call can be said to correspond to a `task'.
Since it returns a token, it becomes possible to specify
task dependencies by refering to a token as a dependency
in a later specified task.
\begin{lstlisting}
queue myQueue;
auto myTokA = myQueue.submit
   ( [&](handler& h) {
       h.parallel_for<class taskA>(...);
     }
   );
auto myTokB = myQueue.submit
   ( [&](handler& h) {
       h.depends_on(myTokA);
       h.parallel_for<class taskB>(...);
     }
   );
\end{lstlisting}

\Level 1 {Race conditions}

Sycl has the same problems with
\emph{race conditions}\index{race condition!in SYCL}
that other shared memory system have:

\cxxverbatimsnippet{dpcppsumreduct}

To get this working correctly would need either
a reduction primitive or atomics on the accumulator.
The 2020 proposed standard has improved atomics.

\cxxverbatimsnippet{dpcppsumreduction}

\Level 1 {Reductions}

Reduction operations were added in the the SYCL 2020 Provisional Standard,
meaning that they are not yet finalized.

Here is one approach, which works in \indexterm{hipsycl}:
%
\cxxverbatimsnippet{syclsumreduct}
%
Here a \lstinline+sycl::+\indexsycldef{reduction} object is created
from the target data and the reduction operator. 
This is then passed to the \indexsyclshow{parallel_for}
and its \indexsycldef{combine} method is called.

\Level 0 {Memory access}

Memory treatment in SYCL is a little complicated, because is (at the very least)
host memory and device memory, which are not necessarily coherent.

There are also three mechanisms:
\begin{itemize}
\item Unified Shared Memory, based on ordinary C/C++ `star'-pointers.
\item Buffers, using the \indexsyclshow{buffer} class;
  this needs the \indexsyclshow{accessor} class to access the data.
\item Images.
\end{itemize}

\begin{table}[ht]
  \caption{Memory types and treatments}
  \label{tab:sycl-mem}  
  \begin{tabular}{llll}
    \toprule
    Location&allocation&coherence&copy to/from device \\
    \midrule
    Host  &\n{malloc}                   &explicit transfer   &\indexsyclshow{queue::memcpy}\\
          &\indexsyclshow{malloc_host}  &coherent host/device&\\
    Device&\indexsyclshow{malloc_device}&explicit transfer   &\indexsyclshow{queue::memcpy}\\
    Shared&\indexsyclshow{malloc_shared}&coherent host/device&\\
    \bottomrule
  \end{tabular}
\end{table}

\Level 1 {Unified shared memory}

Memory allocated with \indexsyclshow{malloc_host}
is visible on the host:
\cxxverbatimsnippet[code/dpcpp/cxx/outshared.cxx]{syclsharealloc}

Device memory is allocated with \indexsyclshow{malloc_device},
passing the queue as parameter:
%
\cxxverbatimsnippet{sycldevcalloc}
%
Note the corresponding \indexsyclshow{free} call
that also has the queue as parameter.

Note that you need to be in a parallel task. 
The following gives a segmentation error:
\begin{lstlisting}
  [&](sycl::handler &cgh) {
    shar_float[0] = host_float[0];
  }
\end{lstlisting}

Ordinary memory, for instance from \indexsyclshow{malloc},
has to be copied in a kernel:
\cxxverbatimsnippet[code/dpcpp/cxx/outdevice.cxx]{sycldevcmemcpy}

\Level 1 {Buffers and accessors}

Arrays need to be declared in a way such that they can be
access from any device.

\cxxverbatimsnippet[code/dpcpp/cxx/forloop.cxx]{syclbufdef}

Inside the kernel, the array is then unpacked from the buffer:

\cxxverbatimsnippet[code/dpcpp/cxx/forloop.cxx]{syclbufuse}

However, the \indexsyclshow{get_access} function results
in a \lstinline+sycl::+\indexsyclshow{accessor}, not a pointer to a simple type.
The precise type is templated and complicated, so this 
is a good place to use \lstinline+auto+.

Accessors can have a mode associated:
\lstinline+sycl::access::mode::+\indexsyclshow{read}
\lstinline+sycl::access::mode::+\indexsyclshow{write}

\begin{dpcppnote}
\begin{lstlisting}
    array<floattype,1> leftsum{0.};
#ifdef __INTEL_CLANG_COMPILER
    sycl::buffer leftbuf(leftsum);
#else
    sycl::range<1> scalar{1};
    sycl::buffer<floattype,1> leftbuf(leftsum.data(),scalar);
\end{lstlisting}
\end{dpcppnote}

\begin{dpcppnote}
there are modes
\begin{lstlisting}
// standard
sycl::accessor acc = buffer.get_access<sycl::access::mode:write>(h);
// dpcpp extension
sycl::accessor acc( buffer,h,sycl::read_only );
sycl::accessor acc( buffer,h,sycl::write_only );
\end{lstlisting}
\end{dpcppnote}

\Level 1 {Querying}

The function \indexsyclshow{get_range} can query the size of either a buffer
or an accessor:
\cxxverbatimsnippet[code/dpcpp/cxx/range2.cxx]{syclbufrange}
\cxxverbatimsnippet[code/dpcpp/cxx/range2.cxx]{syclaccrange}

\Level 0 {Parallel output}

There is a \lstinline+sycl::+\indexsyclshow{cout} and \lstinline+sycl::+\indexsyclshow{endl}.

\cxxverbatimsnippet[code/dpcpp/cxx/hello.cxx]{syclcout}

Since the end of a queue does not flush stdout,
it may be necessary to call
\lstinline+sycl::queue::+\indexsyclshow{wait}
\begin{lstlisting}
myQueue.wait();  
\end{lstlisting}

\Level 0 {DPCPP extensions}

Intel has made some extensions to SYCL:
\begin{itemize}
\item Unified Shared Memory,
\item Ordered queues.
\end{itemize}

\Level 0 {Intel devcloud notes}

\n{qsub -I} for interactive session.

\n{gdb-oneapi} for debugging.

\url{https://community.intel.com/t5/Intel-oneAPI-Toolkits/ct-p/oneapi}
for support.

\Level 0 {Examples}

\Level 1 {Kernels in a loop}

The following idiom works:
%
\cxxverbatimsnippet[code/dpcpp/cxx/jacobi1d.cxx]{syclkernelloop}

\Level 1 {Stencil computations}

The problem with stencil computations is that only interior points are updated.
Translated to SYCL: we need to iterate over a subrange of the range over which
the buffer is defined. First let us define these ranges:
%
\cxxverbatimsnippet[code/dpcpp/cxx/jacobi1d.cxx]{syclrangebc}
%
Note the boundary value~$1.$ on the right boundary.

Restricting the iteration to the interior points is done through
the \indexsyclshow{offset} parameter of the \indexsyclshow{parallel_for}:
%
\cxxverbatimsnippet[code/dpcpp/cxx/jacobi1d.cxx]{sycliteratebc}
