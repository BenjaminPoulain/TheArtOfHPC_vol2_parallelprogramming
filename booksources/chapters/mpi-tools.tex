% -*- latex -*-
%%%%%%%%%%%%%%%%%%%%%%%%%%%%%%%%%%%%%%%%%%%%%%%%%%%%%%%%%%%%%%%%
%%%%%%%%%%%%%%%%%%%%%%%%%%%%%%%%%%%%%%%%%%%%%%%%%%%%%%%%%%%%%%%%
%%%%
%%%% This text file is part of the source of 
%%%% `Parallel Programming in MPI and OpenMP'
%%%% by Victor Eijkhout, copyright 2012-2022
%%%%
%%%% mpi-tools.tex : tools interface
%%%%
%%%%%%%%%%%%%%%%%%%%%%%%%%%%%%%%%%%%%%%%%%%%%%%%%%%%%%%%%%%%%%%%
%%%%%%%%%%%%%%%%%%%%%%%%%%%%%%%%%%%%%%%%%%%%%%%%%%%%%%%%%%%%%%%%

\index{MPI!tools interface|(}

Recent versions of MPI have a standardized way of reading out
performance variables: the \emph{tools interface}
which improves on the old interface described in section~\ref{sec:profile}.

\Level 0 {Initializing the tools interface}
\label{sec:mpitools-init}

The tools interface requires a different initialization routine
\indexmpidef{MPI_T_init_thread}
\begin{lstlisting}
int MPI_T_init_thread( int required,int *provided );
\end{lstlisting}
Likewise, there is \indexmpidef{MPI_T_finalize}
\begin{lstlisting}
int MPI_T_finalize();
\end{lstlisting}
These matching calls can be made multiple times,
after MPI has already been initialized with
\indexmpishow{MPI_Init} or \indexmpishow{MPI_Init_thread}.

Verbosity level is an integer parameter.
\begin{verbatim}
MPI_T_VERBOSITY_{USER,TUNER,MPIDEV}_{BASIC,DETAIL,ALL}
\end{verbatim}

\Level 0 {Control variables}
\label{sec:mpit-cvar}
\index[mpi]{control variable|(}
\index[mpi]{cvar|see{control variable}}

\emph{Control variables} are implementation-dependent variables
that can be used to inspect and/or control the internal workings of MPI.
Accessing control variables requires initializing the tools interface;
section~\ref{sec:mpitools-init}.

We query how many \emph{control variables} are available
with \indexmpiref{MPI_T_cvar_get_num}.
%% \begin{lstlisting}
%% int MPI_T_cvar_get_num( int *number_of_cvars );
%% \end{lstlisting}
%
A description of the control variable can be obtained
from \indexmpiref{MPI_T_cvar_get_info}.
%% \begin{lstlisting}
%% int MPI_T_cvar_get_info( int cvar_num,
%%   char *name, int *name_length,
%%   int *verbosity, MPI_Datatype *type,MPI_T_enum *enumtype,
%%   char *description,int *description_length,
%%   int *bind,int *scope);
%% \end{lstlisting}
\begin{itemize}
\item
  An invalid index leads to a function result of \indexmpidef{MPI_T_ERR_INVALID_INDEX}.
\item
  Any output parameter can be specified as \indextermtt{NULL}
  and MPI will not set this.
\item
  The \lstinline{bind} variable is an object type or \indexmpidef{MPI_T_BIND_NO_OBJECT}.
\item
  The \lstinline{enumtype} variable is \indexmpishow{MPI_T_ENUM_NULL} if the variable
  is not an enum type.
\end{itemize}
\cverbatimsnippet{mpicvarlist}
\begin{remark}
  There is no constant indicating a maximum buffer length for these variables.
  However, you can do the following:
  \begin{enumerate}
  \item Call the info routine with \n{NULL} values for the buffers,
    reading out the buffer lengths;
  \item allocate the buffers with sufficient length, that is,
    including an extra position for the null terminator; and
  \item calling the info routine a second time, filling in the string buffers.
  \end{enumerate}
\end{remark}

Conversely, given a variable name, its index can be retrieved with
\indexmpidef{MPI_T_cvar_get_index}:
\begin{lstlisting}
int MPI_T_cvar_get_index(const char *name, int *cvar_index)
\end{lstlisting}
If the name can not be matched, the index is \indexmpidef{MPI_T_ERR_INVALID_NAME}.

Accessing a control variable is done through a
\indextermbus{control variable}{handle}.
\begin{lstlisting}
int MPI_T_cvar_handle_alloc
   (int cvar_index, void *obj_handle,
    MPI_T_cvar_handle *handle, int *count)
\end{lstlisting}
The handle is freed with \indexmpidef{MPI_T_cvar_handle_free}:
\begin{lstlisting}
int MPI_T_cvar_handle_free(MPI_T_cvar_handle *handle)
\end{lstlisting}

\emph{Control variable access}\index{control variable!access}
is done through \indexmpidef{MPI_T_cvar_read} and \indexmpidef{MPI_T_cvar_write}:
\begin{lstlisting}
int MPI_T_cvar_read(MPI_T_cvar_handle handle, void* buf);
int MPI_T_cvar_write(MPI_T_cvar_handle handle, const void* buf);
\end{lstlisting}

\index[mpi]{control variable|)}

\Level 0 {Performance variables}
\index[mpi]{performance variable|(}
\index[mpi]{pvar|see{performance variable}}

The realization of the tools interface is installation-dependent,
you first need to query how much of the tools interface is provided.

\cverbatimsnippet[code/mpi/c/mpitpvar.c]{mpit_init_q}

\cverbatimsnippet[code/mpi/c/mpitpvar.c]{mpit_list}

\begin{raggedlist}
  Performance variables come in classes:
  \indexmpidef{MPI_T_PVAR_CLASS_STATE}
  \indexmpidef{MPI_T_PVAR_CLASS_LEVEL}
  \indexmpidef{MPI_T_PVAR_CLASS_SIZE}
  \indexmpidef{MPI_T_PVAR_CLASS_PERCENTAGE}
  \indexmpidef{MPI_T_PVAR_CLASS_HIGHWATERMARK}
  \indexmpidef{MPI_T_PVAR_CLASS_LOWWATERMARK}
  \indexmpidef{MPI_T_PVAR_CLASS_COUNTER}
  \indexmpidef{MPI_T_PVAR_CLASS_AGGREGATE}
  \indexmpidef{MPI_T_PVAR_CLASS_TIMER}
  \indexmpidef{MPI_T_PVAR_CLASS_GENERIC}
\end{raggedlist}

Query the number of performance variables with \indexmpidef{MPI_T_pvar_get_num}:
\begin{lstlisting}
int MPI_T_pvar_get_num(int *num_pvar);
\end{lstlisting}
Get information about each variable, by index, with \indexmpidef{MPI_T_pvar_get_info}:
\begin{lstlisting}
int MPI_T_pvar_get_info
   (int pvar_index, char *name, int *name_len,
    int *verbosity, int *var_class, MPI_Datatype *datatype,
    MPI_T_enum *enumtype, char *desc, int *desc_len, int *bind,
    int *readonly, int *continuous, int *atomic)
\end{lstlisting}
See general remarks about these in section~\ref{sec:mpit-cvar}.
\begin{itemize}
\item The \lstinline{readonly} variable indicates that the variable can not be written.
\item The \lstinline{continuous} variable requires use of
  \indexmpishow{MPI_T_pvar_start} and \indexmpishow{MPI_T_pvar_stop}.
\end{itemize}

Given a name, the index can be retried with \indexmpidef{MPI_T_pvar_get_index}:
\begin{lstlisting}
int MPI_T_pvar_get_index(const char *name, int var_class, int *pvar_index)
\end{lstlisting}
Again, see section~\ref{sec:mpit-cvar}.

\Level 1 {Performance experiment sessions}

To prevent measurements from getting mixed up, they need to be done in
\indextermsub{performance experiment}{session}s,
to be called `sessions' in this chapter.
However see section~\ref{sec:session}.

Create a session with \indexmpidef{MPI_T_pvar_session_create}
\begin{lstlisting}
int MPI_T_pvar_session_create(MPI_T_pvar_session *session)
\end{lstlisting}
and release it with \indexmpidef{MPI_T_pvar_session_free}:
\begin{lstlisting}
int MPI_T_pvar_session_free(MPI_T_pvar_session *session)
\end{lstlisting}
which sets the session variable to \indexmpidef{MPI_T_PVAR_SESSION_NULL}.

We access a variable through a handle, associated with a certain session.
The handle is created with \indexmpidef{MPI_T_pvar_handle_alloc}:
\begin{lstlisting}
int MPI_T_pvar_handle_alloc
   (MPI_T_pvar_session session, int pvar_index,
    void *obj_handle, MPI_T_pvar_handle *handle, int *count)
\end{lstlisting}
(If a routine takes both a session and handle argument, and
the two are not associated, an error of \indexmpidef{MPI_T_ERR_INVALID_HANDLE}
is returned.)

Free the handle with \indexmpidef{MPI_T_pvar_handle_free}:
\begin{lstlisting}
int MPI_T_pvar_handle_free
   (MPI_T_pvar_session session,
    MPI_T_pvar_handle *handle)
\end{lstlisting}
which sets the variable to \indexmpidef{MPI_T_PVAR_HANDLE_NULL}.

Continuous variables (see \indexmpishow{MPI_T_pvar_get_info} above, which outputs this)
can be started and stopped with
\indexmpidef{MPI_T_pvar_start} and \indexmpidef{MPI_T_pvar_stop}:
\begin{lstlisting}
int MPI_T_pvar_start(MPI_T_pvar_session session, MPI_T_pvar_handle handle);
int MPI_T_pvar_stop(MPI_T_pvar_session session, MPI_T_pvar_handle handle)
\end{lstlisting}
Passing \indexmpidef{MPI_T_PVAR_ALL_HANDLES} to the stop call
attempts to stop all variables within the session.
Failure to stop a variable returns \indexmpidef{MPI_T_ERR_PVAR_NO_STARTSTOP}.

Variables can be read and written with
\indexmpidef{MPI_T_pvar_read} and \indexmpishow{MPI_T_pvar_write}:
\begin{lstlisting}
int MPI_T_pvar_read
   (MPI_T_pvar_session session, MPI_T_pvar_handle handle,
    void* buf)
int MPI_T_pvar_write
   (MPI_T_pvar_session session, MPI_T_pvar_handle handle,
    const void* buf)
\end{lstlisting}
If the variable can not be written
(see the \lstinline{readonly} parameter of \indexmpishow{MPI_T_pvar_get_info}),
 \indexmpidef{MPI_T_ERR_PVAR_NO_WRITE} is returned.

A special case of writing the variable is to reset it with
\begin{lstlisting}
int MPI_T_pvar_reset(MPI_T_pvar_session session, MPI_T_pvar_handle handle)  
\end{lstlisting}
The handle value of \indexmpishow{MPI_T_PVAR_ALL_HANDLES} is allowed.

A call to \indexmpishow{MPI_T_pvar_readreset}
is an atomic combination of the read and reset calls:
\begin{lstlisting}
int MPI_T_pvar_readreset
   (MPI_T_pvar_session session,MPI_T_pvar_handle handle, 
    void* buf)
\end{lstlisting}

\index[mpi]{performance variable|)}

\Level 0 {Categories of variables}

Variables, both the control and performance kind,
can be grouped into categories by the MPI implementation.

The number of categories is queried with
\indexmpidef{MPI_T_category_get_num}:
\begin{lstlisting}
int MPI_T_category_get_num(int *num_cat)
\end{lstlisting}
and for each category the information is retrieved with
\indexmpidef{MPI_T_category_get_info}:
\begin{lstlisting}
int MPI_T_category_get_info
   (int cat_index,
    char *name, int *name_len, char *desc, int *desc_len, 
    int *num_cvars, int *num_pvars, int *num_categories)
\end{lstlisting}
For a given category name the index can be found with
\indexmpidef{MPI_T_category_get_index}:
\begin{lstlisting}
int MPI_T_category_get_index(const char *name, int *cat_index)
\end{lstlisting}

The contents of a category are retrieved with
\indexmpidef{MPI_T_category_get_cvars},
\indexmpidef{MPI_T_category_get_pvars},
\indexmpidef{MPI_T_category_get_categories}:
\begin{lstlisting}
int MPI_T_category_get_cvars(int cat_index, int len, int indices[])
int MPI_T_category_get_pvars(int cat_index, int len, int indices[])
int MPI_T_category_get_categories(int cat_index, int len, int indices[])
\end{lstlisting}

\begin{raggedlist}
  These indices can subsequently be used in the calls
  \indexmpishow{MPI_T_cvar_get_info},
  \indexmpishow{MPI_T_pvar_get_info},
  \indexmpishow{MPI_T_category_get_info}.
\end{raggedlist}

If categories change dynamically, this can be detected with
\indexmpidef{MPI_T_category_changed}
\begin{lstlisting}
int MPI_T_category_changed(int *stamp)
\end{lstlisting}

\Level 0 {Events}

\cverbatimsnippet{mpiteventcount}
\cverbatimsnippet{mpiteventinfo}

\index{MPI!tools interface|)}

